\section{L4Re}
\label{sec:l4re}

Kernkonzept entwickelt das \textit{L4Re} Mikrokernel-Betriebssystem.
Die Entwicklung begann als Forschungsprojekt an der Professor für
Betriebssysteme der TU Dresden. Das Unternehmen wurde von ehemaligen/noch an
der Universität arbeitenden wissenschaftlichen Mitarbeitern gegründet. L4Re
wird zudem weiterhin zu Lehrzwecken an der Universität eingesetzt.

L4Re setzt sich aus dem Mikrokernel der dritten Generation Fiasco.OC und einer
zugehörigen Laufzeitumgebung zusammen. Zwar ist die Sicherheit von Fiasco.OC im
Gegensatz zu seL4 nicht formal verifiziert, allerdings existiert im Moment
keine mit L4Re vergleichbare vollständige Laufzeitumgebung für seL4, welches
also nur eingeschränkt in kommerziellen Anwendungen nutzbar ist.

Teile des L4Re Quellcodes sind frei zugänglich und stehen unter Open-Source
Lizenzen. Eingesetzt wird L4Re generell überall dort, wo die zertifizierbare
Sicherheit des Systems einen Vorteil gegenüber einer reinen Linux-Lösung o.ä.
bietet. Das System ist für eine Reihe von Hardware-Architekturen (x86, ARM,
MPIS, ...) und Mikrocontroller-Plattformen verfügbar.
