\section{Die L4Re Programmschnittstelle}
\label{sec:l4re_pi}

Der folgende Abschnitt beschreibt kurz die Entwicklung eines L4Re Programmes
und geht dabei insbesondere auf \textit{Capabilities} und
\textit{Interprozesskommunikation (IPC)} rein.

\textit{Capabilities} sind ein zentraler L4Re Mechanismus. Eine Capability ist
ein Pointer auf ein vom Kernel verwaltes Objekt. Der Kernel verwaltet pro
Task\footnote{Die Begriffe \textit{Task} und \textit{Thread} tragen im L4Re
Kontext nicht dieselben Bedeutungen wie z.B. unter Linux:
\begin{itemize}
  \item[\textbf{Task}]
    Abstrahiert einen virtuellen Adressraum und den Zugriff auf Kernel-Objekte
    über Capability Index Arrays.
  \item[\textbf{Thread}]
    Ausführungsstrang, enkapsuliert Ausführungszustand, CPU Zustand,
    Scheduling Parameter etc. Threads können an Tasks gebunden werden und
    erhalten über diese Zugriff auf deren virtuellen Adressraum und Capabilities.
    Dabei ist es Threads möglich, sich zur Laufzeit von einem Task loszulösen
    und sich an einen anderen Task zu binden.
  \item[\textbf{Programm}]
    Alleinstehende L4Re Anwendungen die aus einem Task und mehreren Threads
    bestehen können.
\end{itemize}}
ein Array von Capability-Indizes. Jeder Eintrag in diesem Array referenziert
ein Kernel Objekt und über Angabe des zugehörigen Indizes kann der Task dessen
Funktionalität nutzen, vorausgesetzt die Capability besitzt ausreichende
Rechte.

Der Datenaustausch zwischen einem Task und einem Kernel Objekt
geschieht dabei über einen \textit{User Thread Control Block (UTCB)}. UTCBs
sind Speicherbereiche die vom Kernel für jeden an einen Task gebundenen
Thread bereitgestellt werden.

Die vielleicht wichtigsten Kernel Objekte sind \textit{IPC Gates}. Diese
implementieren Kommunikationskanäle zur Interprozesskommunikation zwischen
Threads, welche von einem Thread gelesen und von mehreren nebenläufig
beschrieben werden können. Über IPCs können dabei sowohl Daten als auch
Resourcen ausgetauscht werden. Capabilities werden dabei per \textit{Flexpages}
übertragen. Diese beschreiben Regionen im Adressraum des sendenden Threads
zusammen mit Typ- und Zugriffsrecht-Informationen. Über diesen Mechanismus
können Capabilities ,,weitergereicht'' werden (aber jeweils nur mit gleichen
oder weiter eingeschränkten Rechten). Z.B. ist es dadurch möglich mittels
\textit{Dataspace} Capabilities geteilten Speicher zwischen unterschiedlichen
L4Re Programmen zu realisieren.

Die Nutzung von Capabilities zur Kommunikation zwischen zwei L4Re Programmp
soll nun anhand der Beispiel-Anwendung \texttt{l4re\_ipc\_example}
veranschaulicht werden. Listing~\ref{lst:l4re_ipc_example_tree} zeigt die für
L4Re Anwendungen typische Paket-Struktur von \texttt{l4re\_example}.

\begin{mintlisting}[label=lst:l4re_ipc_example_tree]{minttext}{\texttt{l4re-ipc-example} Verzeichnisstruktur}
l4re-ipc-example
├── configs
│   ├── l4re_ipc_example.cfg
│   └── modules.list
├── server
│   ├── src
│   │   ├── l4re_ipc_example_common.c
│   │   ├── l4re_ipc_example_common.h
│   │   ├── l4re_ipc_example_receiver.c
│   │   ├── l4re_ipc_example_sender.c
│   │   └── Makefile
│   └── Makefile
└── Makefile
\end{mintlisting}

Die Anwendung besteht aus zwei Programmen, \texttt{l4re\_ipc\_example\_sender} und
\texttt{l4re\_ipc\_example\_receiver}. Das sendende Programm überträgt per IPC einen
beliebigen Block Text an das empfangende Programm, welches ihn nach
\texttt{stdout} ausgibt.

Die beiden Programme werden zur Laufzeit durch das \textit{Ned Skript}
\texttt{l4re\_ipc\_example.cfg} gestartet und miteinander verknüpft. Ned ist
ein L4Re Programm, welches es ermöglicht, mittels Lua Skripten Capabilities zu
erzeugen, sowie L4Re Programme zu starten, denen diese Capabilities über ihr
ursprüngliches \textit{Environment} zur Verfügung gestellt werden können.
Listing~\ref{lst:l4re_ipc_example_ned} zeigt den Quelltext dieses Ned Skripts.
In diesem wird zunächst die IPC Gate Capability \texttt{chan\_cap} erzeugt.
Dann werden Sender- und Empfänger-Programm gestartet und in diese jeweils
\texttt{chan\_cap} unter dem Namen \texttt{"cap"} hereingereicht.

\begin{mintlisting}[label=lst:l4re_ipc_example_ned]{mintlua}{\texttt{l4re\_ipc\_example.cfg}}
local l4 = require 'L4'
local l = l4.default_loader

chan_cap = l:new_channel()
chan_capname = 'cap'

l:start({ caps = { [chan_capname] = chan_cap },
          log = { 'snd', 'blue' } },
        'rom/l4re_ipc_example_sender ' .. chan_capname)

l:start({ caps = { [chan_capname] = chan_cap:svr() },
          log = { 'rcv', 'green' } },
        'rom/l4re_ipc_example_receiver ' .. chan_capname)
\end{mintlisting}

Die Datei \texttt{modules.list}
beschreibt die zur Ausführung der Beispiel-Applikation notwendigen Programme und
Skripte wie in Listing~\ref{lst:l4re_ipc_example_modules_list} gezeigt.

\begin{mintlisting}[label=lst:l4re_ipc_example_modules_list]{minttext}{\texttt{modules.list}}
/* 1 */ entry l4re-example
/* 2 */ kernel fiasco
/* 3 */ roottask moe rom/l4re_ipc_example.cfg
/* 4 */ module l4re
/* 5 */ module ned
/* 6 */ module l4re_ipc_example_sender
/* 7 */ module l4re_ipc_example_receiver
/* 8 */ module l4re_ipc_example.cfg
\end{mintlisting}

Zeile 2 gibt hier den Namen der zu verwendenden Kernel-Binärdatei an. Zeile 3
definiert das nach dem Booten von Fiasco auszuführende L4Re-Programm, dies ist
in fast allen Fällen \texttt{Moe}. Moe ist der L4re \textit{Root-Task} der u.a.
den gesamten den User-Space Programmen zur Verfügung stehenden Arbeitsspeicher
verwaltet.  Zusätzlich werden die Pakete \texttt{l4re} und \texttt{ned}
benötigt. Ersteres ist der L4Re-Kernel, letzteres implementiert die in
\texttt{l4re\_ipc\_example.cfg} verwendete Lua-API.  Schließlich folgen die
Programme selbst sowie das Ned-Skript \texttt{l4re\_example.cfg}. Man beachte,
dass letzteres bereits in Zeile 3 auftaucht, dies ist ein Hinweis an Moe, nach
der Initialisierung grundlegender System-Funktionen dieses Skript mittels der
von Ned implementierten API auszuführen.

Listings~\ref{lst:l4re_ipc_example_sender}
und~\ref{lst:l4re_ipc_example_receiver} zeigen den (gekürzten) Quellcode des
Sender- und Empfänger-Programms. Man beachte, dass hier der Einfachheit halber
mit Funtionen aus dem C Interface von L4Re gearbeitet wird. Das Packet
\texttt{l4re-core}, welches den Großteil dieser Funktionen bereitstellt, ist
selbst überwiegend in C++ geschrieben, biete aber auch eine C API deren Nutzung
an dieser Stelle deutlicher macht, was im Detail passiert.

\begin{mintlisting}[label=lst:l4re_ipc_example_sender]{mintc}{\texttt{l4re\_ipc\_example\_sender.c}}
/* text to be sent to receiving server */
static char str[] = "Hello world";

int
main(int argc, char **argv)
{
  assert(argc == 2);

  printf("Hello from sender\n");

  /* get IPC capability */
  l4_cap_idx_t chan = l4re_env_get_cap(argv[1]);

  /* create dataspace */
  l4re_ds_t ds = l4re_util_cap_alloc();

  l4re_ma_alloc(sizeof(str), ds, 0);

  /* copy text into dataspace */
  l4_addr_t *addr = 0;
  l4re_rm_attach((void **)&addr,
                 sizeof(str),
                 L4RE_RM_SEARCH_ADDR,
                 ds,
                 0,
                 L4_PAGESHIFT);

  memcpy(addr, str, sizeof(str));

  /* send IPC message */
  l4_fpage_t fpage = l4_obj_fpage(ds, 0, L4_FPAGE_RWX).raw;

  l4_msg_regs_t *mr = l4_utcb_mr();
  mr->mr[0] = l4_map_obj_control(0, 0);
  mr->mr[1] = fpage;

  l4_msgtag_t send_tag = l4_msgtag(IPC_LABEL, 0, 1, 0);

  l4_ipc_send(chan, l4_utcb(), send_tag, L4_IPC_NEVER),

  l4_sleep(100);

  /* free allocated ressources */

  /* ... */

  printf("Goodbye from sender\n");

  exit(EXIT_SUCCESS);
}
\end{mintlisting}

\begin{mintlisting}[label=lst:l4re_ipc_example_receiver]{mintc}{\texttt{l4re\_ipc\_example\_receiver.c}}
int
main(int argc, char **argv)
{
  assert(argc == 2);

  printf("Hello from receiver\n");

  /* get IPC capability */
  l4_cap_idx_t chan = l4re_env_get_cap(argv[1]);

  /* bind capability to main thread */
  l4_rcv_ep_bind_thread(chan, l4re_env()->main_thread, IPC_THREAD_LABEL);

  /* receive IPC */
  l4re_ds_t ds = l4re_util_cap_alloc();

  l4_buf_regs_t *br = l4_utcb_br();
  br->bdr = 0;
  br->br[0] = ds | L4_RCV_ITEM_SINGLE_CAP | L4_RCV_ITEM_LOCAL_ID;

  l4_msgtag_t tag;
  l4_umword_t label;
  l4_msgtag_t tag = l4_ipc_wait(l4_utcb(), &label, L4_IPC_NEVER);

  if (label != IPC_THREAD_LABEL)
    printf("Incorrect thread label\n");

  if (l4_msgtag_label(tag) != IPC_LABEL)
    printf("Incorrect tag label\n");

  /* dump content of received dataspace */
  l4_addr_t *addr = 0;
  l4re_rm_attach((void **)&addr,
                 l4re_ds_size(ds),
                 L4RE_RM_SEARCH_ADDR,
                 ds,
                 0,
                 L4_PAGESHIFT);

  printf("Received: %s\n", (char *)addr);

  printf("Goodbye from receiver\n");

  exit(EXIT_SUCCESS);
}
\end{mintlisting}
