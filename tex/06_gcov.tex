\section{Code-Coverage}

Um die Auswertung von Unit-Tests zu verbessern, sollte automatisierte
Line-Coverage in das existierende Test-System integriert werden.

Das L4Re Build-System unterstützt hauptsächlich den \textit{GNU C Compiler
(gcc)}.  Mit diesem lassen sich Code-Coverage Daten erzeugen, indem zu
analysierende Programme mit speziellen Flags kompiliert und gelinkt werden.
Diese sorgen dafür, dass zunächst in den im Kompilierungs-Schritt erzeugten
Object-Files Maschinen-Code eingebaut wird, der bei Ausführung dafür sorgt,
dass u.a. die Anzahl der Abarbeitungen für jede nicht heraus-optimierte
Quellcode-Zeile in speziellen Datenstrukturen festgehalten wird. Außerdem
werden die Programme gegen zusätzlichen Code gelinkt, der vor Ausführung von
\texttt{main} einen Exit-Handler registriert. In diesem werden gegen Ende des
Programms die in einer statisch allokierten verketteten Liste (mit einem
Listen-Element für jedes Object-File aus dem das ausgeführte Programm erzeugt
wurde) gespeicherten Coverage-Daten in ein binäres Datenformat umgewandelt und
im Dateisystem neben den Quellcode-Dateien, aus denen das Programm erzeugt
wurde, abgelegt. Genau genommen werden pro Quellcode-Datei zwei für die
Generierung von Coverage-Dateien relevante Dateien erzeugt: eine Datei im
\texttt{.gcno} Format wird bereits während des Compilierungs-Schrittes
generiert. Nach Programmausführung entsteht wie beschrieben zusätzlich eine
\texttt{.gcda} Binärdatei. Beide werden benötigt um menschenlesbare
Coverage-Reports zu generieren.

Im Zusammenhang mit L4Re ist hier problematisch, dass die derartige Erzeugung
von Coverage-Daten das Vorhandensein eines Dateisystems voraussetzt.  Diese
Bedingungen ist für die in einer virtuellen Maschine ausgeführten Unit-Tests
nicht gegeben. Vielmehr wäre es wünschenswert, die Daten auf der seriellen
Konsole auszugeben, sodass sie bei Testausführung gegebenenfalls von einem
Wrapper-Skript geparst und im Dateisystem des Host-Systems abgelegt werden
können. Mit dem intern zur Test-Ausführung verwendeten Perl 5 Programm
\textit{Tapper} existierte bereits ein solches Wrapper-Skript, welches bisher
aber nur die bei Ausführung von Tests auf der seriellen Konsole ausgegebenen
Ergebnisse parsen und für den Nutzer in ein ,,angenehmeres'' Format
transformieren konnte.

Die Funktionen zur Ausgabe von Coverage Daten sind in gcc in
\texttt{libgcc/gcov.h} und \texttt{gcc/gcov*} implementiert. Sie lassen sich
durch eine eigene Implementierung ersetzen, indem eine mit
\texttt{-fprofile-arcs} und \texttt{-ftest-coverage} kompilierte Objekt-Datei
gegen eine solche gelinkt wird. Listing~\ref{lst:gcov} zeigt den
,,alternativen'' gcov Exit-Handler, welcher die Coverage-Daten mittels
\texttt{printf} ausgibt anstatt sie in eine Datei zu schreiben. Die Marker
\texttt{GCOV GCDA/DATA/DONE} können später von Tapper erkannt werden. Tapper
schreibt dann die im Hexadezimal-Format ausgegebenen Binärdaten in die
angegebenen \texttt{.gcda} Datei im Host-Dateisystem.

\begin{mintlisting}[label=lst:gcov]{mintc}{Modifizierter gcov Exit-Handler}
static void __attribute__((destructor))
gcov_exit(void)
{
  /* buffer to hold extracted gcda data */
  enum { Buf_sz = 1 << 15 };

  static char output_buffer[Buf_sz];

  /* iterate over the list of coverage data nodes */
  struct gcov_info *tmp = __gcov_master.list_start;

  while (tmp)
    {
      /* dump gcda path */
      printf("GCOV GCDA %s\n", gcov_info_filename(tmp));

      /* dump coverage data */
      size_t buf_sz = convert_to_gcda(NULL, tmp);
      assert(buf_sz < Buf_sz);

      convert_to_gcda(output_buffer, tmp);

      printf("GCOV DATA ");
      for (size_t i = 0; i < buf_sz; ++i)
        printf("%02X", (output_buffer[i] & 0xff));
      putchar('\n');

      /* continue with the next coverage data node */
      tmp = tmp->next;
    }

  printf("GCOV DONE\n");
}
\end{mintlisting}
