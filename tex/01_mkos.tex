\section{Mikrokernel-Betriebssysteme}
\label{sec:mkos}

\subsection{Grundidee}

Mikrokernel sind Betriebssystem-Kernel die lediglich minimale, zur Konstruktion
eines vollständigen Betriebssystems nötige, Funktionalität implementieren, z.B.
Adressräume, Threads und Interprozesskommunikation.

Im Gegensatz zu \textit{monolithischen} Betriebssystemen (wie zum Beispiel
GNU/Linux) werden klassischerweise vom privilegierten Kernel ausgeführte
Grundfunktionen des Systems (z.B. Geräte-Treiber) in User-Space Programme
ausgelagert. Das Grundprinzip hinter modernen Mikrokerneln lautet hierbei
\textit{,,capability, not policy''}: Der Kernel stellt idealerweise lediglich
die zur Nutzung der unterliegenden Hardware nötige minimale Funktionalität
bereit, trifft aber keine strategischen Entscheidungen. Dies wird stattdessen
von einer Reihe von User-Space Programmen übernommen\footnote{Diese Darstellung
ist eine Idealisierung, in der Praxis kann es Sinn ergeben, besonders
Performance-kritische Funktionalität wie z.B.  Scheduling in den Kernel zu
integrieren auch wenn dies das genannte Prinzip verletzt.}

Vorteil ist dabei die verbesserte Stabilität und Sicherheit des Systems.  Ein
kompliziertes System wie Linux mit vielen Millionen Code-Zeilen ist sehr
anfällig für unentdeckte Fehler, welche die Stabilität des Systems gefährden
oder zum Beispiel von einem Angreifer ausgenutzt werden können, um
privilegierte Programme auszuführen.  Die Sicherheit eines Mikrokernels, der in
nur einigen zehntausend Zeilen Code implementiert ist, kann dagegen mit
realistischem Aufwand sehr genau überprüft werden.  Selbst wenn es einem
Angreifer gelingt, ein User-Space Programm zu kompromittieren, bleibt zudem die
Integrität des Kernels unberührt und der potentielle Schaden ist deutlich
geringer da das entsprechende Programm im Bedarfsfall beendet und neu gestartet
werden kann und ohnehin als User-Space Programm nur indirekten Hardware-Zugriff
hat.

Ein weiterer Vorteil ist die Modularität des Systems, User-Space Komponenten
mit identischen Schnittstellen können (sogar zur Laufzeit) untereinander
ausgetauscht werden (vergleichbar mit heute weitverbreiteten
\textit{Microservices}) und das gesamte System lässt sich potentiell einfacher
auf verschiedene physische Maschinen verteilen.

Problematisch ist unter Umständen die Performance des Systems: Mikrokernel
machen viel größeren Gebrauch von Interprozesskommunikation als monolithische
Kernel und diese ist zudem oft synchron. Eine der Herausforderungen bei der
Entwicklung eines Mikrokernel-Systems ist die bestmögliche Bewältigung solcher
Performance-Hürden.

\subsection{Historische Entwicklung}

Die Grundidee hinter Mikrokernel-Betriebssystemen existiert bereits seit den
60er Jahren~\cite{rc4000}. Frühe Mikrokernel, z.B. Mikrokernel Varianten von
\textit{Mach}~\cite{mach} litten vor allem unter Performance Problemen bei der
Interprozesskommunikation.

Großer Fortschritt in diesem Bereich gelang Jochen Liedtke.  Dieser stellte die
These auf, dass schlechte Performance kein inhärentes Problem des
Mikrokernel-Ansatzes, sondern das Resultat suboptimaler Implementierungen war
und entwickelte \textit{L3}~\cite{l3} in dem Interprozesskommunikation über 20
mal schneller war als in Mach. L3 war dabei noch kompakter als vorangegangene
Mikrokernel.

Mit \textit{L4}~\cite{l4} wurde später ein weiterer Mikrokernel entwickelt,
dessen Hauptziel hohe Performance war und der einige weitere in Mach präsente
Probleme umging. Damit wurde die Ära der kompakten und performanten Mikrokernel
der \textit{zweiten Generation} eingeleitet. Es folgten eine Reihe von von L4
basierten Systemen, z.B. \textit{L4Ka::Pistachio}, \textit{L4/MIPS},
\textit{L4/Fiasco} und \textit{OKL4}.

Spätere Mikrokernel der \textit{dritten Generation} legten dann den Fokus auf
\textit{Capabilities}, Virtualisierung und Verifizierbarkeit.

Capabilities sind dabei ein Mechanismus zur Zugriffs-Kontrolle.  Um auf eine
Ressource (z.B. Speicher) zuzugreifen, muss ein Programm eine Capability auf
diese besitzen.  Capabilities können zwischen Programmen ausgetauscht werden,
z.B. kann der Erzeuger einer Ressource Zugriff auf diese an eine Reihe von
Verbraucher weiterleiten. Eine Capability ist sowohl ein Zeiger auf als auch eine
Zugriffs-Berechtigung für einer Ressource.  Zugriffs-Verwaltung lässt sich somit
in den User-Space auslagern. Dieses Prinzip ist deutlich eleganter als
Zugriffs-Verwaltung unter UNIX-ähnlichen Systemen, in denen der Kernel bei
jedem Zugriff auf eine Ressource die Berechtigungen des ausführenden Programmes
überprüfen muss.

Capabilities waren bereits in Mikrokerneln der ersten Generation, z.B. Mach
implementiert aber aus Performance-Gründen kein Teil mehr von Mikrokerneln der
zweiten Generation.  Diese verwendeten alternative Ansätze die sich im
Nachhinein als suboptimal entpuppten. Mikrokernel der dritten Generation
implementieren Capabilities auf effizientere Art und Weise, die es gleichzeitig
erlaubt, die Verfügbarkeit von Ressourcen besser zu verfolgen.

Manche Mikrokernel der dritten Generation werden häufig als Hypervisor (z.B.
für ,,unsichere'' monolithische Systeme wie GNU/Linux) eingesetzt, ein Beispiel
ist \textit{Fiasco.OC}~\cite{fiasco}. Da dabei vor allem (sowohl akademisches
als auch kommerzielles) Interesse an der Sicherheit solcher Systeme besteht,
ist Verifizierbarkeit ein aktueller Trend. Mit \textit{seL4}~\cite{sel4}
existiert so z.B. ein Mikrokernel, dessen Sicherheit (ausgehend von gewissen
Axiomen) mathematisch bewiesen ist.

Neben der L4 Familie existieren heute viele weitere ganz oder teilweise den
Mikrokernel-Ansatz verfolgende Betriebssysteme. Beispiele sind das Andrew S.
Tanenbaum ursprünglich als Lehrsystem entwickelte \textit{Minix}, der von
Linux unabhängige \textit{GNU Hurd} Kernel und Googles \textit{Fuchsia/Zircon}.
