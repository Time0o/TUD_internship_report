\section{uvmm}
\label{sec:uvmm}

uvmm ist ein auf L4Re basierter \textit{Virtual Machine Monitor/Hypervisor}.
Eine uvmm Instanz ist ein gewöhnliches L4Re User-Space Programm, welches ein
virtualisiertes Gast-System ausführt. Dabei ermöglicht uvmm das Ausführen von
Gast-Images, die für die gleiche Architektur gebaut wurden, auf der uvmm
ausgeführt wird; Unterstützt werden bisher x86, ARM und MIPS.

uvmm mappt dabei den gesamten physischen Speicher eines Gastes in seinen
virtuellen Adressraum und vermittelt alle Interaktionen zwischen dem Gast und
der Hardware. U.a. existiert mit \textit{L4virtio} eine Implementierung des
\textit{Virtio} Standards für L4Re die schnellen Hauptspeicher-Zugriff und hohe
Netzwerk-Performance durch Kooperation von Treibern und Hypervisor (in diesem
Fall uvmm) ermöglicht.

Eine klassische Anwendungsmöglichkeit von uvmm ist das Ausführen mehrerer
von der Hardware und teils oder ganz voneinander isolierten Linux Instanzen in
sicherheitskritischen Anwendungen.
